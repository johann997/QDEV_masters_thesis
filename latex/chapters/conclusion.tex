\chapter{Conclusion}\label{cha:conclusion}


\epigraph{Nothing in life is to be feared, it is only to be understood. Now is the time to understand more, so that we may fear less.}{Marie Curie}

In conclusion, a novel method has been demonstrated to study the conductance enhancement from the Kondo effect, in a more weakly coupled regime than has been previously explored. 
Previous studies have predominantly focused on very strong coupling, where the Kondo temperature exceeds the system temperature, even between conductance peaks.
Weaker coupling reveals a markedly different behavior: the Kondo temperature is significantly lower than the system temperature between conductance peaks, resulting in a lack of conductance enhancement.
Nonetheless, as the dot energy approaches the Fermi energy of the source and drain leads, the Kondo temperature increases exponentially . 
Consequently, even under relatively weak coupling, a small conductance enhancement persists on the shoulder of odd occupation conductance peaks.




To reliably measure this small conductance enhancement, a new method is needed to disentangle effects of charge motion, entropy and Kondo enhancement on the shape and location of the conductance. 
The approach relies on two key components. 
The first component is a simultaneous measurement of a conductance and charge transition. The charge transition is then used to determine the occupation of the quantum dot. This occupation is used as a reference to remove the effects of charge motion and entropy. When conductance is plotted against occupation, a shift in the maximum conduction to higher occupation suggests Kondo enhancement. 
The second component, is measuring conductance at range of temperatures that cross into the temperature broadened regime $\mathrm{\Gamma/T} \lesssim 1$. A global fit to NRG including each temperature set point is used to reliably determine $\mathrm{\Gamma/T}$ and leverarm. The determined $\mathrm{\Gamma/T}$ is used to compare the corresponding NRG with data to test agreement. 

Agreement between data and corresponding NRG is seen at coupling strengths $\mathrm{\Gamma/T} = 20.5$, $\mathrm{\Gamma/T} = 9.7$ and $\mathrm{\Gamma/T} = 2.5$. The most strongly coupled data ($\mathrm{\Gamma/T} = 20.5$) shows the greatest shift in conductance peak towards higher occupation. 
A reliable measurement of the charge transition is critical to determine the occupation of the quantum dot. 
Hence, various current set points through the charge sensor QPC were tested to ensure independence from the sensor's settings.
Despite differences in the underlying charge transition shape across current set points, agreement between conductance versus occupation data and NRG was consistently observed.
To investigate the influence of coupling symmetry on Kondo enhancement, four ratios of coupling symmetry were measured.
Good agreement with NRG was found with symmetric coupling. 
However, a shift towards greater occupation than predicted by NRG was observed as coupling symmetry became more asymmetric.
This discrepancy suggests a difference between the $\mathrm{\Gamma/T}$ value determined from the global fits to conductance and that which best matches the corresponding charge transitions. 
To validate this observation, separate global fits to conductance and charge transitions were performed. Under symmetric coupling, the determined $\mathrm{\Gamma/T}$ values agree. However, as asymmetry increases, the $\mathrm{\Gamma/T}$ obtained from the charge transitions is significantly larger than that from the conductance fits.

Only one other experiment has investigated the effects of coupling symmetry~\cite{kondo_asymmetric}. 
It was found that the characteristic zero bias peak in between two conductance peaks shifted to nonzero bias. 
Although conducted in a more weakly coupled regime, further exploration of bias dependence in the asymmetrically coupled regime could potentially recover the discrepancy with NRG predictions. 
Additionally, whilst previous studies have not used a charge sensor for precise charge transition measurements alongside conductance, some have utilised it as a noise source to examine the controlled dephasing of the Kondo singlet~\cite{kondo_controlled_dephasing}. 
However, these experiments applied a bias across the charge sensor twelve times larger, resulting in an effective $6\%$ decrease in conductance. 
This resulting effect is too small to explain the nearly twofold difference in $\mathrm{\Gamma/T}$ determined by global fits to conductance and charge transitions.
Finally, although no experiment can measure conductance when the quantum dot is coupled to a single lead. 
A previous study observed the change in entropy as an electron entered a quantum dot with similar coupling strengths as those in this thesis~\cite{child_strong}. 
A suppression of the entropy was reported, however, the expected shift in the  entropy onset was not seen in the data. 
In contrast, the measurements on asymmetric coupling in this thesis found the conductance maximum shifts towards greater occupation than predicted by NRG.
A device capable of measuring entropy alongside conductance with two leads, holds promise for illuminating this discrepancy.


The disagreement between comparisons to NRG with symmetric and asymmetric coupling is interesting. However, this data was taken in one cooldown at a single dot setting. A further cooldown is required to verify this discrepancy.