\chapter{Conclusion}\label{cha:conclusion}


\epigraph{``Nothing in life is to be feared, it is only to be understood. Now is the time to understand more, so that we may fear less.''}{Marie Curie}

\noindent In conclusion, a method has been demonstrated to study the conductance enhancement from the Kondo effect, in a more weakly coupled regime than has been previously explored. 
Previous studies have predominantly focused on very strong coupling, where the Kondo temperature exceeds the system temperature, even between Coulomb peaks.
Weaker coupling reveals a markedly different behaviour: the Kondo temperature is significantly lower than the system temperature between Coulomb peaks, resulting in a lack of conductance enhancement.
Nonetheless, as the dot energy approaches the Fermi energy of the source and drain leads, the Kondo temperature increases exponentially. 
Consequently, even under relatively weak coupling, a small conductance enhancement persists on the shoulder of odd occupation Coulomb peaks.




To reliably measure this small conductance enhancement, a new method is required to disentangle the effects of charge motion, entropy, and Kondo enhancement on the shape and location of the conductance. 
The approach relies on two key components. 
The first component is a simultaneous measurement of a conductance and charge transition. The charge transition is then used to determine the occupation of the quantum dot. This occupation is used as a reference to remove the effects of charge motion and entropy. When conductance is plotted against the occupation, a shift in the maximum conduction to higher occupation suggests Kondo enhancement. 
The second component is measuring conductance at a range of temperatures that cross into the temperature broadened regime, $\mathrm{\Gamma/k_BT} \lesssim 1$. A global fit to NRG, including each temperature setpoint is used to reliably determine $\mathrm{\Gamma/k_BT}$ and lever arm. The determined $\mathrm{\Gamma/k_BT}$ is then used to compare data with NRG to test agreement. 

Agreement between data and corresponding NRG is seen at coupling strengths $\mathrm{\Gamma/k_BT} = 20.5$, $\mathrm{\Gamma/k_BT} = 9.7$ and $\mathrm{\Gamma/k_BT} = 2.5$. The most strongly coupled data ($\mathrm{\Gamma/k_BT} = 20.5$) shows the greatest shift in conductance maximum towards higher occupation. 
A reliable measurement of the charge transition is critical to determine the occupation of the quantum dot. 
Hence, various current setpoints through the charge sensor were tested to ensure independence from the charge sensor's settings.
Despite differences in the underlying charge transition shape across current setpoints, agreement with NRG was consistently observed.
To investigate the influence of coupling symmetry on Kondo enhancement, four ratios of coupling symmetry were measured.
Good agreement with NRG was found with symmetric coupling. 
However, a shift towards greater occupation than predicted by NRG was observed as coupling symmetry became more asymmetric.
This discrepancy suggests a difference between the $\mathrm{\Gamma/k_BT}$ value determined from the global fits to conductance and that which best matches the corresponding charge transitions. 
To validate this observation, separate global fits to conductance and charge transitions were performed. Under symmetric coupling, the determined $\mathrm{\Gamma/k_BT}$ values agree. However, as asymmetry increases, the $\mathrm{\Gamma/k_BT}$ obtained from the charge transitions is significantly larger than that from the conductance fits.
The discrepancy with NRG is surprising as conversations with our theorist collaborators suggest that NRG only depends on the overall $\Gamma$, and not the ratio between $\Gamma_\mathrm{L}$ and $\Gamma_\mathrm{R}$.

To my knowledge, only one other experiment has investigated the effects of coupling symmetry on the Kondo effect~\cite{kondo_asymmetric}. 
It was found that the characteristic zero bias peak between Coulomb peaks shifted to a nonzero bias. However, this effect would require a strong energy dependence of the tunnel barriers (i.e., $\mathrm{\Gamma = \Gamma(E)}$) in a NRG calculation, which is not assumed in our current NRG calculations.

An interesting insight into our observed discrepancy comes from very recent theoretical studies of the differential conductance through a quantum dot and its dependence on asymmetry in the tunnel barriers and source-drain bias~\cite{Tsutsumi2021,kondo_nrg_asymmetric}. It was found that higher-order corrections to the differential conductance include non-linear terms that depend strongly on the tunnelling and bias asymmetries. However, corrections to the differential conductance in this paper require an applied bias across the quantum dot. As the measurements in this thesis were conducted in an effective zero bias limit (\qty{1}{\micro V} bias), it is unclear whether the discrepancy can be explained by this theory. A further cooldown would allow for a more focused exploration of tunnel coupling symmetry, bias symmetry and total applied bias. If our data qualitatively matches these recent theory predictions, a new calculation of our NRG, including bias and tunnel coupling symmetry may be required. 
 

Finally, a recent study observed the change in entropy as an electron entered a quantum dot (coupled to a single lead) with similar coupling strengths as those in this thesis~\cite{child_strong}. 
A suppression of the entropy was reported, however, the expected shift in the entropy onset was not seen in the data. 
In contrast, the measurements on asymmetric coupling in this thesis found the conductance maximum shifts towards greater occupation than predicted by NRG.
Perhaps a device capable of measuring entropy and conductance with two leads holds promise for illuminating this discrepancy.

