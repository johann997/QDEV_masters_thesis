\chapter{Conclusion}\label{cha:conclusion}


In conclusion, a novel method has been demonstrated to study the conductance enhancement of the Kondo effect in a more weakly coupled regime than has been previously measured in other studies. The Kondo temperature drops exponentially as the dot energy fall below the energy level of the source and drain leads. In the less strongly coupled regime, the Kondo temperature can drop below the system temperature on the shoulder of the conductance. This results in very small enhancement of the conductance from the Kondo effect.

To reliably measure the conductance enhancement, a new method was needed to isolate effects of charge motion, entropy and Kondo enhancement on the shape and location of the conductance. The first ingredient of our method is a simultaneous measurement of conductance and charge transition. This charge transition is used to determine the occupation of the quantum dot. The occupation reference is used to remove the effects of charge motion and entropy. When conductance is plotted against occupation, a shift in the maximum conduction to higher occupation suggests Kondo enhancement. The second ingredient is measuring conductance at range of temperatures that cross into the temperature broadened regime $\mathrm{\Gamma/T} \lesssim 1$ . A global fit to NRG with each temperature setpoint is used to reliably determine $\mathrm{\Gamma/T}$ and leverarm. $\mathrm{\Gamma/T}$ is used to plot the corresponding conductance versus occupation NRG ontop of the data to test agreement. 

Agreement between data and corresponding NRG is seen at coupling strengths $\mathrm{\Gamma/T} = 20.5$, $\mathrm{\Gamma/T} = 9.7$ and $\mathrm{\Gamma/T} = 2.5$. Where the most strongly coupled data ($\mathrm{\Gamma/T} = 20.5$) shows the greatest conductance maximum shift to higher occupation. As a simultaneous measurement of the charge transition is critical to determine the occupation. Various current setpoints through the charge sensor QPC were tested to check for dependence on the charge sensor QPC. Although the underlying charge transition shape differed between current setpoints through the charge sensor QPC, agreement was found between conductance versus occupation and NRG at each setpoint. To test if Kondo enhancement changes when the quantum dot is coupled to two reservoirs versus a single reservoir, four ratios of coupling symmetry were measured.  Good agreement with NRG was found with symmetric coupling. However, as the coupling became more asymmetric, the maximum conductance shifted to greater occupation than NRG. This suggests a discrepancy between the $\mathrm{\Gamma/T}$ determined from the global fit to conductance and the $\mathrm{\Gamma/T}$ which best matches the corresponding charge transition. This is tested with separate global fits to the conductance and charge transition. At symmetric coupling, the determined $\mathrm{\Gamma/T}$ agree. However, as asymmetry increases, the $\mathrm{\Gamma/T}$ from the global fit to charge transitions is much larger than $\mathrm{\Gamma/T}$ from the global fit to conductance. 