\chapter[Occupation Resolved Conductance of the Kondo Effect]{Occupation Resolved Conductance of the Kondo Effect}\label{cha:mixed_valence_conductance}
% Probably should be called: Occupation resolved conductance of the Kondo effect


To my knowledge, the following work in this chapter is the first time that conductance and charge transitions have been simultaneously measured to resolve the conductance enhancement due to the Kondo effect as function of the occupation in an experiment. The conductance enhancement from Kondo is most pronounced when the quantum dot is strongly coupled ($\mathrm{\Gamma/k_BT}>>1$) to the source and drain leads. Previous measurements of the Kondo effect have focused on the conductance enhancement in the middle of Coulomb blockade valleys. However, when the coupling is strong, charge transitions spread out and they become difficult to convert into occupation. The following work falls between the tightly defined window of strong coupling to see the Kondo effect and weak coupling where charge transitions are easily measured. 

\section{Introduction}
Previous measurements of the Kondo effect in quantum dots have studied the temperature dependence of conductance in the middle of the Coulomb blockade valley where the theory is well understood~\cite{kondo_unitary, costi_kondo_mv_eo_regime}. In the Kondo-regime, the Kondo temperature Eq.~\ref{eq:kondo_temp}, sets a new many body scale and is used to universally scale the conductance Eq.~\ref{eq:kondo_conductance}. This one parameter scaling of the conductance in the Kondo-regime is used as one of the demonstrations of the Kondo effect. When the coupling between the quantum dot and source and drain reservoirs is reduced, the Kondo temperature drops below the system temperature and the conductance enhancement is not seen in the middle of Coulomb blockade valley. Few experiments have studied this the region of parameter space~\cite{goldhaber_mv}. Such studies observed the parameter $\mathrm{s}$ Eq.~\ref{eq:kondo_conductance} and found that in the Kondo-regime far from charge degeneracy, $\mathrm{s}$ was constant and equal to $0.20$. However, $\mathrm{s}$ varied rapidly as $\tilde{\epsilon}_0>-0.5$ which is when the dot energy $\epsilon_0$ is within one half of the coupling strength away from the source and drain reservoir energy levels.  

It was our aim to measure the Kondo-effect with close to zero conductance in the Coulomb blockade valley between conductance peaks. In such as regime the conductance enhancement due to Kondo is very small and only occurs on the shoulder of the conductance peak. As the dot energy is further lowered, the Kondo temperature plummets and the conductance enhancement ends. As the conductance enhancement is very small, a measurement of conductance only could not be used to investigate the Kondo effect in this regime for two reasons. Firstly, entropy shifts the occupation of the quantum dot with temperature. This effect is taken advantage of in recently developed entropy measurement techniques~\cite{hartman, child_strong, child_meas}. This shift in occupation can be ignored when coupling is strong and conductance enhancement in the Coulomb blockade valley is large. In this regime changes in occupation with respect to gate voltage are small. Secondly, charge motion in the dopant layer can shift the conductance left or right with respect to the gate voltage. This shifting due to charge motion renders it impossible to retro actively shift the conductance to offset the effects of entropy. 

For these two reasons, it is necessary to simultaneously measure a second 'reference' signal alongside the conductance. This reference signal can then be used to offset the above effects and allow for comparison of the conductance across multiple temperatures. We measure the charge in the quantum dot alongside conductance. We assume a linear relation between the measured change in charge and the real occupation of the quantum dot. The occupation is used to understand the conductance as a function of occupation. 



\section{Fitting Conduction and Charge Transitions to NRG}
To investigate the enhancement of conductance due to the Kondo effect with intermediate coupling, the conductance and charge of the quantum dot are simultaneously measured. A simultaneous measurement of the charge and conductance is trivial on the hardware side. Two different current amplifiers are connected to the device to measure the conductance through the quantum dot and current through the charge sensor. Each current amplifier is connected to two different analog-to-digital converters (ADCs), which sample data points simultaneously. It is much trickier to tune the quantum dot into a regime where both the conductance and charge transition can be used for comparison to theory. When weakly coupled ($\mathrm{\Gamma/k_BT}<<1$), the charge transitions have a sharp drop in current and can be fit analytically Eq.~\ref{eq:cs_lineshape}. However, conductance becomes sharply peaked and the conductance amplitude drops with strength of coupling, 

\begin{equation}\label{eq:cond_amp}
  \mathrm{G_0} = 
  \frac
  {\Gamma_\mathrm{L}\cdot\Gamma_\mathrm{R}}
  {\Gamma_\mathrm{L} + \Gamma_\mathrm{R}}
\end{equation}

Where $\mathrm{G_0}$ is the conductance maximum. With weak coupling, the conductance enhancement due to Kondo is negligible and the drop in conductance amplitude can  lead to the signal being lost in the noise floor. 
%NEED TO IMPROVE THIS WORDING.

With stronger coupling, the Kondo temperature increases leading to measurable enhanced conductance. However, strongly coupled charge transitions become very broadened Fig.~\ref{fig:ch1/virtual_gate_example}. Hence, strong coupling requires large sweeps in gate voltage to cover the range where the quantum dot is un-occupied to fully occupied. Due to the cross-capacitive coupling between the sweep gate and the charge sensor, the change in current through the charge sensor can be pushed into a non-linear regime. A non-linear relationship between the current through the charge sensor and addition of charge into the quantum dot makes an extraction of dot occupation from the charge transition unreliable. Hence, a virtual gate that keeps the current through the charge sensor constant is required Fig.~\ref{fig:ch1/virtual_gate_example}. The exact ratio of gates used to form the virtual gate can drastically change the underlying shape of the charge transition, this is discussed in an upcoming section.   


\subsection{Numerical Renormalisation Group (NRG)}
The dot is carefully tuned to a coupling regime where enhancement of the conductance due to Kondo is expected and charge transitions are reliably measured. As previous tests of the Kondo effect are not possible in this regime, we rely on comparison to Numerical Renormalisation Group (NRG) calculations~\cite{nrg}. These calculations are provided by our theory collaborators (Yigal Meir, Yaakov Kleeorin, and Andrew Mitchell). 
We received two 2d datasets corresponding to conductance and occupation. The columns correspond energy scaled by $\Gamma$ (energy/$\Gamma$) so that the x-axis is unit-less. Each row in the NRG corresponds to a different value of $\mathrm{\Gamma/T}$.  We confirmed with Yaakov that the lineshape does not change if both $\mathrm{T}$ and $\mathrm{\Gamma}$ increase together, that is, the lineshape really represents the ratio $\mathrm{\Gamma/T}$ not $\mathrm{\Gamma}$ or $\mathrm{T}$ individually.
To compare NRG to our data, we need to pull off the row corresponding to the correct $\mathrm{\Gamma/T}$ and also determine the correct leverarm which scales the NRG x-axis to be in units of mV. In addition to this, both the conductance and occupation NRG is scaled by an amplitude and x-offset term. The occupation NRG has extra parameters, a y-offset which is the current through the charge sensor, a linear term which is the cross-capacitance between the sweep gate and charge sensor and an occupation dependant linear term as we noticed a change in cross-capacitance from the additional charge of an electron in the quantum dot. Many of these parameters have little cross-correlation when fitting the data to NRG, meaning the global minimum of the minimiser is reliably reached. However, changes to $\mathrm{\Gamma/T}$ can be offset by a change to the leverarm and fitting a single trace allowing both parameters to freely vary is unreliable. 

\begin{figure}[!bht]
  \begin{center}
%% includegraphics: comment the following if not using the graphicx package
    \includegraphics[width=0.8\textwidth]{figures/ch3/crop_FiguresMaster.013.png}
    \caption[Method to determine gamma and leverarm and fit to charge transitions]{\label{fig:ch3/cond_ct_gf} 
    % For some options that work with pdf\LaTeX, please see this discussion:
    %   \url{http://tex.stackexchange.com/questions/11839}.  
    (\textbf{a}) Conductance data as a single electron enters a strongly coupled ($\mathrm{\Gamma/k_BT > 1}$) quantum dot at four different temperatures. The x location of the conductance maxima has not been shifted, it is the original x location of the acquired data. The fits (in grey) are from a global fit to NRG, where the gamma and leverarm parameters are held fixed across all four temperatures. (\textbf{b}) Charge transitions are measured simultaneously with the conductance. The charge transitions are fit to NRG where the gamma and leverarm parameters are held fixed to the values determined from the global fit to conductance.}
  \end{center}
\end{figure}

\subsection{Fitting Conductance to NRG}
In the temperature broadened regime $\mathrm{\Gamma/T} < 1$ and leverarm are decoupled as the broadening of the conductance or charge transition is from temperature only. To access the temperature broadened regime when $\mathrm{\Gamma/T} > 1$ the temperature of the fridge is increase until $\mathrm{\Gamma/T} \lesssim 1$. Data is taken at multiple temperature setpoints across this range so that  $\mathrm{\Gamma/T}$ and leverarm can be reliably determined Fig.~\ref{fig:ch3/cond_ct_gf}. Conductance and charge transitions are simultaneously measured at each temperature. However, it is the conductance data that is used to determine $\mathrm{\Gamma/T}$ and leverarm as it is in general more clean. A global fit to the conductance data including each of the temperature setpoints is used Fig.~\ref{fig:ch3/cond_ct_gf}\textbf{a}. Where $\mathrm{\Gamma}$ and leverarm are allowed to vary but held fixed between temperatures and $\mathrm{T}$ is held fixed to the calculated electron temperature at each fridge temperature Fig.~\ref{fig:ch1/electron_temp}. The other parameters used to fit the NRG to conductance (amplitude and x-offset) are allowed to freely vary.
The fitting range is chosen to be the full width at $90\%$ the maximum conductance. This removes any bias of picking a 'good' fitting range. 

\subsection{Fitting Charge Transitions to NRG}
The global fit to conductance is used to determine $\mathrm{\Gamma/T}$ and leverarm, which are then used in the fits to the charge transitions. Each charge transition is fit separately where the value of $\mathrm{\Gamma/T}$ and leverarm from the conductance global fits are held fixed. All other parameters (amplitude, x-offset, y-offset, linear and occupation dependant linear) are allowed to freely vary Fig.~\ref{fig:ch3/cond_ct_gf}\textbf{b}. The fitting range of the charge transitions is maximised up-to but excluding charge jumps.


\subsection{Determining Occupation from Charge Transitions}

\begin{figure}[!bht]
  \begin{center}
    \includegraphics[width=0.8\textwidth]{figures/ch3/crop_FiguresMaster.014.png}
    \caption[Method to determine occupation and plot conductance vs. occupation]{\label{fig:ch3/cond_vs_occ_gf} 
    % For some options that work with pdf\LaTeX, please see this discussion:
    %   \url{http://tex.stackexchange.com/questions/11839}.  
    (\textbf{a}) Charge transitions are converted to occupation by removing the relevant fit parameters. These are amplitude, current offset, cross capacitance of virtual gate and occupation dependant cross capacitance. (\textbf{b}) A plot of conductance vs. occupation is used to show the enhanced conductance due to Kondo. As temperature decreases, the conductance maxima occurs at higher occupation. The NRG (grey) conductance vs. occupation corresponding to the determined $\mathrm{\Gamma/k_BT}$ is plotted ontop of the data where good agreement is found at each temperature.}
  \end{center}
\end{figure}

The charge transition cannot be used as a reference to compare with conductance. However, the charge transition can be turned into an occupation which can be used as a reference to compare conductance enhancement between dot settings. The fit parameters from the NRG fit to the charge transition are used to remove relevant terms. The current offset (y-offset) and linear term are removed trivially. The occupation dependant linear term is removed using by multiplying this linear term with the correct NRG occupation row corresponding to $\mathrm{\Gamma/k_BT}$. Finally, the charge transition is scaled by the amplitude. The corresponding NRG at the correct $\mathrm{\Gamma/k_BT}$ only has to be shifted by the x-offset as it is already in units of occupation Fig.~\ref{fig:ch3/cond_vs_occ_gf}\textbf{a}.


\subsection{Conductance versus Occupation Varying Temperature}





\section{Conductance versus Occupation Varying Coupling}

\begin{figure}[!bht]
  \begin{center}
%% includegraphics: comment the following if not using the graphicx package
    \includegraphics[width=0.8\textwidth]{figures/ch3/crop_FiguresMaster.015.png}
    \caption[Conductance vs. Occupation : Varying the coupling strength between the quantum dot and leads]{\label{fig:ch3/cond_occ_couplingstrength} 
    % For some options that work with pdf\LaTeX, please see this discussion:
    %   \url{http://tex.stackexchange.com/questions/11839}.  
    Conductance vs. occupation in a weak (green) and strong (blue) coupling regime. Each trace is taken at \qty{20}{mK}. The coupling strength $\mathrm{\Gamma/k_BT}$ was determined from a global fit to multiple temperatures. The NRG (grey) conductance vs. occupation corresponding to the determined $\mathrm{\Gamma/k_BT}$ is plotted ontop of the data where good agreement is found at each coupling strength.}
  \end{center}
\end{figure}



\section{Varying Charge Sensor Current}

\subsection{Charge Transition Dependence on Charge Sensor Current}

\begin{figure}[!bht]
  \begin{center}
%% includegraphics: comment the following if not using the graphicx package
    \includegraphics[width=0.8\textwidth]{figures/ch3/crop_FiguresMaster.016.png}
    \caption[Charge transitions measured at various current set points through the charge sensor]{\label{fig:ch3/cond_occ_ct_set-points} 
    % For some options that work with pdf\LaTeX, please see this discussion:
    %   \url{http://tex.stackexchange.com/questions/11839}.  
    Current through the charge sensor (green) from pinch off to the start of the first conductance plateau. Corresponding charge transition (yellow) at each of the current set-points through the charge sensor. Note, the x-axis is not the same between the underlying QPC trace and charge transitions. The charge transitions x-axis has been scaled the same amount for clarity. As the current through the charge sensor is changed, the charge transitions vary dramatically. The left and right slopes curve upwards, downwards, in the same direction or opposite to each other.}
  \end{center}
\end{figure}



\subsection{Conductance versus Occupation Varying Charge Sensor Current}

\begin{figure}[!bht]
  \begin{center}
%% includegraphics: comment the following if not using the graphicx package
    \includegraphics[width=0.8\textwidth]{figures/ch3/crop_FiguresMaster.017.png}
    \caption[Conductance vs. Occupation : Varying the current through the charge sensor]{\label{fig:ch3/cond_occ_QPC_vs_ct} 
    % For some options that work with pdf\LaTeX, please see this discussion:
    %   \url{http://tex.stackexchange.com/questions/11839}.  
    (\textbf{a}) Charge transitions measured with high (blue) and low (green) current through the charge sensor. The transition are offset in current for ease of comparison. The x-axis uses the same virtual gate for each charge transition. The slopes on either side of the charge transitions vary with current through the charge sensor indicating the optimal virtual gate also changes. (\textbf{b}) Conductance vs. occupation ($\mathrm{\Gamma/k_BT=21}$) at different currents through the charge sensor. The traces are offset for clarity. Each trace is taken at \qty{20}{mK}. The coupling strength $\mathrm{\Gamma/k_BT}$ was determined from a global fit to multiple temperatures. The NRG (grey) conductance vs. occupation corresponding to the determined $\mathrm{\Gamma/k_BT}$ is plotted on-top of the data where good agreement is found for each current set-point.}
  \end{center}
\end{figure}


\section{Varying Coupling Symmetry}

\subsection{Determining Coupling Symmetry Ratio}

\begin{figure}[!bht]
  \begin{center}
%% includegraphics: comment the following if not using the graphicx package
    \includegraphics[width=0.8\textwidth]{figures/ch3/crop_FiguresMaster.018.png}
    \caption[Conductance vs. Occupation : Picking locations of varying coupling symmetry]{\label{fig:ch3/symmetry_picking} 
    % For some options that work with pdf\LaTeX, please see this discussion:
    %   \url{http://tex.stackexchange.com/questions/11839}.  
     (\textbf{a}) A 2d scan of conductance through the quantum dot varying the two coupling gates V\textsubscript{CSS} and V\textsubscript{RC}. In the top left, the dot is more coupled to the right reservoir than the left $\mathrm{\Gamma_R} = 0.05\cdot\mathrm{\Gamma_L}$. In the middle of the scan (where the conductance is maximum), the coupling is symmetric $\mathrm{\Gamma_R} = \mathrm{\Gamma_L}$ (\textbf{b}) Charge transitions measured with different ratios of coupling between the two leads in the dot. $\mathrm{\Gamma_R/\Gamma_L} = 1.0$ is symmetric coupling (pink), $\mathrm{\Gamma_R/\Gamma_L} = 0.05$ is asymmetric coupling (green). The transition are offset in current for ease of comparison, but have roughly the same current set-point through the QPC. The x-axis uses the same virtual gate for each charge transition. 
     % The charge transitions become more strongly coupled as asymmetric increases (pink-green)
    }
  \end{center}
\end{figure}



\section{Conductance versus Occupation Varying Coupling Symmetry}



\begin{figure}[!bht]
  \begin{center}
%% includegraphics: comment the following if not using the graphicx package
    \includegraphics[width=0.8\textwidth]{figures/ch3/crop_FiguresMaster.019.png}
    \caption[Conductance vs. Occupation : Varying the coupling symmetry between quantum dot and leads]{\label{fig:ch3/cond_occ_assymetry} 
    % For some options that work with pdf\LaTeX, please see this discussion:
    %   \url{http://tex.stackexchange.com/questions/11839}.  
    (\textbf{a}) Conductance vs. occupation ($\mathrm{\Gamma/k_BT=21}$) with different ratios of coupling between the two leads in the dot. As the asymmetric is increased, the conductance decreases.
    (\textbf{b}) Same data as in (\textbf{a}), except the traces are offset and scaled for clarity. Symmetric coupling agrees well with NRG calculations, however, the asymmetric coupling data is shifted to the right of predicted NRG. This suggests that $\mathrm{\Gamma/k_BT}$ extracted from the global fit to conductance is lower than expected. }
  \end{center}
\end{figure}


\section{Discussion}